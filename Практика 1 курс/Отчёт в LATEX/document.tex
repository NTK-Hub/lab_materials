\documentclass[a4paper,12pt]{article} %размер бумаги устанавливаем А4, шрифт 12пунктов
\usepackage[T2A]{fontenc}
\usepackage[utf8]{inputenc}	%кодировка
\usepackage[english,russian]{babel}%используем русский и английский языки с переносами
\usepackage{amssymb,amsfonts,amsmath,cite,enumerate,float,indentfirst} %пакеты расширений
\usepackage[dvips]{graphicx} %вставка графики
\graphicspath{{images/}}%путь к рисункам

\makeatletter
\renewcommand{\@biblabel}[1]{#1.} % Заменяем библиографию с квадратных скобок на точку:
\makeatother

\usepackage{geometry} % Меняем поля страницы
\geometry{left=3cm}% левое поле
\geometry{right=1.5cm}% правое поле
\geometry{top=2cm}% верхнее поле
\geometry{bottom=2cm}% нижнее поле

\renewcommand{\theenumi}{\arabic{enumi}}% Меняем везде перечисления на цифра.цифра
\renewcommand{\labelenumi}{\arabic{enumi}}% Меняем везде перечисления на цифра.цифра
\renewcommand{\theenumii}{.\arabic{enumii}}% Меняем везде перечисления на цифра.цифра
\renewcommand{\labelenumii}{\arabic{enumi}.\arabic{enumii}.}% Меняем везде перечисления на цифра.цифра
\renewcommand{\theenumiii}{.\arabic{enumiii}}% Меняем везде перечисления на цифра.цифра
\renewcommand{\labelenumiii}{\arabic{enumi}.\arabic{enumii}.\arabic{enumiii}.}% Меняем везде перечисления на цифра.цифра

\newcommand{\imgh}[3]{\begin{figure}[h]\center{\includegraphics[width=#1]{#2}}\caption{#3}\label{ris:#2}\end{figure}}

\begin{document}
	\begin{titlepage}
	\newpage
	
	\begin{center}
		МИНИСТЕРСТВО НАУКИ И ВЫСШЕГО ОБРАЗОВАНИЯ РОССИЙСКОЙ ФЕДЕРАЦИИ
		
		\vspace{1em}
		
		ФЕДЕРАЛЬНОЕ ГОСУДАРСТВЕННОЕ БЮДЖЕТНОЕ ОБРАЗОВАТЕЛЬНОЕ УЧРЕЖДЕНИЕ ВЫСШЕГО ОБРАЗОВАНИЯ
		
		\vspace{1em}
		
		«БЕЛГОРОДСКИЙ ГОСУДАРСТВЕННЫЙ ТЕХНОЛОГИЧЕСКИЙ УНИВЕРСИТЕТ им. В. Г. ШУХОВА» (БГТУ им. В. Г. Шухова)
		 \\
	\end{center}
	
	\vspace{8em}
	
	\begin{center}
		\Large Кафедра программного обеспечения вычислительной техники и автоматизированных систем \\ 
	\end{center}
	
	\vspace{2em}
	
	\begin{center}
		\textsc{\textbf{КОМПЬЮТЕРНАЯ ПРАКТИКА \linebreak}}
	\end{center}
	
	\vspace{6em}
	
	
	
	\newbox{\lbox}
	\savebox{\lbox}{\hbox{Пупкин Иван Иванович}}
	\newlength{\maxl}
	\setlength{\maxl}{\wd\lbox}
	\hfill\parbox{11cm}{
		\hspace*{5cm}\hspace*{-5cm}Выполнил студент:\hfill\hbox to\maxl{Игнатьев Артур\hfill}\\
		\hspace*{5cm}\hspace*{-5cm}Проверил преподаватель:\hfill\hbox to\maxl{Новожён}\\
		\\
		\hspace*{5cm}\hspace*{-5cm}Группа:\hfill\hbox to\maxl{ПВ-223}\\
	}
	
	
	\vspace{\fill}
	
	\begin{center}
		Белгород \\2023
	\end{center}
	
\end{titlepage}% это титульный лист
	\input{RefProject-Description}% это описание
	\input{RefProject-Algoritm}% это описание алгоритмов
	\input{RefProject-Finish}% заключение
	\input{RefProject-App}% приложение
	\newpage
	\tableofcontents % это оглавление, которое генерируется автоматически
\end{document}